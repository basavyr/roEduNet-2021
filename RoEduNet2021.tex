\documentclass[conference]{IEEEtran}

\IEEEoverridecommandlockouts

\usepackage{cite}
\usepackage{amsmath,amssymb,amsfonts}
\usepackage{algorithmic}
\usepackage{graphicx}
\usepackage{textcomp}
\usepackage{xcolor}
\usepackage{hyperref}

\begin{document}

\title{Implementation of an email-based alert system for large-scale system resources}

\author{\IEEEauthorblockN{Robert Poenaru}
\IEEEauthorblockA{\textit{Department of Computational Physics and Information Technology} \\
\textit{\textit{Horia Hulubei} National Institute of Nuclear Physics and Engineering}\\
Magurele, Romania \\
robert.poenaru@protonmail.ch}
}

\maketitle

\begin{abstract}
Tackling the current problems of interest for physicists that deal with various topics require lots of computing simulations. Identifying and preventing any unusual behavior within the system resources that execute large-scale calculations is a crucial process when dealing with system administration, since it can improve the run-time performance of the resources themselves and also help the physicists by obtaining the required results faster. In the present work, a simple \emph{pythonic} implementation which 1) monitors a given computing architecture (i.e., its system resources such as CPU and Memory usage), and 2) alerts a custom team of administrators via e-mail in almost real-time when certain thresholds are passed, is presented. Using existing packages written in Python, with the current implementation it is possible to send e-mails to a predefined list of clients containing detailed information about any machine running outside the "normal" parameters.
\end{abstract}

\begin{IEEEkeywords}
python, system resources, alerting, email, smtp, monitoring, watchdog
\end{IEEEkeywords}

\section{Introduction}

The computing resources within a physics department must be up to speed and ready for a continuous run-time of small-, medium-, but also large-scale simulations, in order to assure a consistent and optimal workflow for the research teams that require calculations. Usually, there is a cohesive workflow between the scientists that want to run their simulations and the system administration (sysadmins) team that provides the necessary resources for executing them. The sysadmins must check that the resources which are performing calculations behave well, but they must also take care of the computing equipment that is sitting in \emph{idle}-mode, in case new requests for allocating resources come from the scientists.
\par However, due to the large degree of complexity of the underlying computing infrastructure, issues related to memory bandwidth, network stability, CPU throttling, cache availability and so on are highly probable, especially when the machines are running continuously. Frequent updates, unexpected network traffic, errors within the services running in the application layer

\subsection{Figures and Tables}
\paragraph{Positioning Figures and Tables} Place figures and tables at the top and 
bottom of columns. Avoid placing them in the middle of columns. Large 
figures and tables may span across both columns. Figure captions should be 
below the figures; table heads should appear above the tables. Insert 
figures and tables after they are cited in the text. Use the abbreviation 
``Fig.~\ref{fig}'', even at the beginning of a sentence.

\begin{table}[htbp]
\caption{Table Type Styles}
\begin{center}
\begin{tabular}{|c|c|c|c|}
\hline
\textbf{Table}&\multicolumn{3}{|c|}{\textbf{Table Column Head}} \\
\cline{2-4} 
\textbf{Head} & \textbf{\textit{Table column subhead}}& \textbf{\textit{Subhead}}& \textbf{\textit{Subhead}} \\
\hline
copy& More table copy$^{\mathrm{a}}$& &  \\
\hline
\multicolumn{4}{l}{$^{\mathrm{a}}$Sample of a Table footnote.}
\end{tabular}
\label{tab1}
\end{center}
\end{table}

\begin{figure}[htbp]
\centerline{\includegraphics{figs/fig1.png}}
\caption{Example of a figure caption.}
\label{fig}
\end{figure}

Figure Labels: Use 8 point Times New Roman for Figure labels. Use words 
rather than symbols or abbreviations when writing Figure axis labels to 
avoid confusing the reader. As an example, write the quantity 
``Magnetization'', or ``Magnetization, M'', not just ``M''. If including 
units in the label, present them within parentheses. Do not label axes only 
with units. In the example, write ``Magnetization (A/m)'' or ``Magnetization 
\{A[m(1)]\}'', not just ``A/m''. Do not label axes with a ratio of 
quantities and units. For example, write ``Temperature (K)'', not 
``Temperature/K''.

\section*{Acknowledgment}

The preferred spelling of the word ``acknowledgment'' in America is without 
an ``e'' after the ``g''. Avoid the stilted expression ``one of us (R. B. 
G.) thanks $\ldots$''. Instead, try ``R. B. G. thanks$\ldots$''. Put sponsor 
acknowledgments in the unnumbered footnote on the first page.

\section*{References}

Please number citations consecutively within brackets \cite{ibm}.



\bibliographystyle{unsrt}

\bibliography{references}

% \begin{thebibliography}{00}
% \bibitem{b1} G. Eason, B. Noble, and I. N. Sneddon, ``On certain integrals of Lipschitz-Hankel type involving products of Bessel functions,'' Phil. Trans. Roy. Soc. London, vol. A247, pp. 529--551, April 1955.

%  @misc{ibm corporation_2012, title={Cpu throttling}, url={https://www.ibm.com/docs/en/tivoli-monitoring/6.3.0?topic=configuration-cpu-throttling}, journal={IBM}, author={IBM Corporation}, year={2012}} 

% \end{thebibliography}
\end{document}
